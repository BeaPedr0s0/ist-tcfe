
\section{Simulation Analysis}
\label{simulation}

\subsection{Operating Point Analysis}
First of all, to contextualize the values obtained using the tools in ngspice, it is necessary to state that, as node 0 is connected to ground, its nodal voltage does not appear on the table of results. Furthermore, to be able to describe the voltage flowing in the dependent source, an extra voltage source (whose voltage is equal to 0 V) was created, and put after the resistor 6 and node 6. This source led to the appearence of node 8, that has the same voltage of node 6. This was necessary because ngspice does not measure the current between two nodes, only in resistors and in independent voltage sources. The circuit with these changes is shown in the drawing below.

\begin{figure}[ht] \centering
\includegraphics[width=1.0\linewidth]{simdraw.pdf}
\caption{Circuit analysed in ngspice.}
\label{simdraw}
\end{figure}


After running the simulation, the results were put in the table below. Then, a careful analysis of the aforementioned table was conducted.It shows the simulated operating point results for the circuit that is being studied, allowing the group to obtain the current flowing in every risistor, the voltage in the dependent voltage source and even the current flowing in the dependent currrent source. 

A variable preceded by @ is of type {\em current} and expressed in Ampere; other variables are of type {\it voltage} and expressed in
    Volt.
\begin{table}[ht]
  \centering
  \begin{tabular}{|l|r|}
    \hline    
    {\bf Name} & {\bf Value [A or V]} \\ \hline
    @cb[i] & 0.000000e+00\\ \hline
@ci[i] & 0.000000e+00\\ \hline
@cout[i] & 0.000000e+00\\ \hline
@q1[ib] & 2.132105e-05\\ \hline
@q1[ic] & 3.676881e-03\\ \hline
@q1[ie] & -3.69820e-03\\ \hline
@q1[is] & 1.945288e-12\\ \hline
@q2[ib] & 3.261073e-04\\ \hline
@q2[ic] & 4.602851e-02\\ \hline
@q2[ie] & -4.63546e-02\\ \hline
@q2[is] & -1.94801e-12\\ \hline
@r1[i] & 5.473928e-04\\ \hline
@r2[i] & 5.260718e-04\\ \hline
@rc[i] & 3.350773e-03\\ \hline
@re[i] & 3.698202e-03\\ \hline
@rin[i] & 0.000000e+00\\ \hline
@rl[i] & 0.000000e+00\\ \hline
@rout[i] & -4.63546e-02\\ \hline
base & 1.052144e+00\\ \hline
coll & 1.947680e+00\\ \hline
emit & 3.698202e-01\\ \hline
emit2 & 2.729077e+00\\ \hline
in & 0.000000e+00\\ \hline
in2 & 0.000000e+00\\ \hline
out & 0.000000e+00\\ \hline
vbe & 6.823234e-01\\ \hline
vcc & 1.200000e+01\\ \hline
vce & 1.577860e+00\\ \hline
veb & 7.813967e-01\\ \hline
vec & 2.729077e+00\\ \hline

  \end{tabular}
  \caption{NgSpice Results}
  \label{tab:op}
\end{table}


In order to validate the results obtained in NGSPICE, relative errors between the theoretical values, obtained in octave and the ones obtained in ngspice, were calculated. These were put in the table below.

\begin{table}[ht] \centering
\begin{tabular}{|
>{\columncolor[HTML]{FFCC67}}l |c|}
\hline
\multicolumn{2}{|l|}{\cellcolor[HTML]{EABD8B}Relative Errors (\%)} \\ \hline
{\color[HTML]{333333} V1}               & 4.444e-05               \\ \hline
{\color[HTML]{333333} V2}               & 4.288e-05               \\ \hline
{\color[HTML]{333333} V3}               & 7.892e-05               \\ \hline
{\color[HTML]{333333} V4}               & 5.154e-05               \\ \hline
{\color[HTML]{333333} V5}               & 0                       \\ \hline
{\color[HTML]{333333} V6}               & 0                       \\ \hline
{\color[HTML]{333333} V7}               & 2.487e-05               \\ \hline
{\color[HTML]{333333} IA}               & 9.253e-05               \\ \hline
{\color[HTML]{333333} IB}               & 1.769e-05               \\ \hline
{\color[HTML]{333333} IC}               & 0                       \\ \hline
\end{tabular}
\caption{Relative Errors between Octave and NgSpice results}
\end{table}




After the anaylysis of these erros, we conclude that the accuracy is extremely high. The absolute maximum error is 9.253e-05\%, which is extremely low. Therefore, the simulation results can be validated.


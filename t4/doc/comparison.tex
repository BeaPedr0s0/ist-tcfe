\section{Comparison}
\label{section:comparison}

\par In this section, a comparison between Octave and Ngspice results will be made. Firstly, the operating point was computed using the theoretical DC model. In the tables below, the three current values ($I_{A}$, $I_{B}$, $I_{C}$) are presented as well as the voltage in the coll.

\begin{table}[ht]
\parbox{.45\linewidth}{
  \centering
  \begin{tabular}{|l|r|}
    \hline    
    {\bf Calculus} & {\bf Value [A or V]} \\ \hline
    @c1[i] & 0.000000e+00\\ \hline
@gb[i] & -2.26065e-04\\ \hline
@r1[i] & 2.161572e-04\\ \hline
@r2[i] & -2.26065e-04\\ \hline
@r3[i] & -9.90741e-06\\ \hline
@r4[i] & 1.183330e-03\\ \hline
@r5[i] & -2.26065e-04\\ \hline
@r6[i] & -9.67173e-04\\ \hline
@r7[i] & 9.671730e-04\\ \hline
v(1) & 5.068716e+00\\ \hline
v(2) & 4.843672e+00\\ \hline
v(3) & 4.369060e+00\\ \hline
v(5) & 4.874693e+00\\ \hline
v(6) & 5.579017e+00\\ \hline
v(7) & -1.98076e+00\\ \hline
v(8) & -2.97458e+00\\ \hline
v(9) & -1.98076e+00\\ \hline

  \end{tabular}
  \caption{Operating point using DC model. Variables are expressed in Ampere or Volt. (Ngspice)}} 
\parbox{.45\linewidth}{
 \centering
  \begin{tabular}{|l|r|}
    \hline    
    {\bf Name} & {\bf Value [A or V]} \\ \hline
    \input{../mat/ponto1_TAB}
  \end{tabular}
  \caption{Operating point using DC model. Variables are expressed in Ampere or Volt.(Octave)}}
\end{table}

As one may observe, some discrepancies are noticeable. These may be due to....


SEI LAAAAAAA FALTA ISTOOOOOOOOOOOOOOOOOOOOOOOOOOO

Moreover, the importance of this calculations must be highlighted. In fact, the theoretical gain expression is dependent on the value of the current $I_{C}$. This relation can be understood by the expression below. Since this incremental parameter is also present in the gain expression, this may be one of the reasons why some discrepancies may be observed when comparing Octave and Ngspice gain resuts. 

\begin {equation}
	 MERIT = \frac{$I_{C}$}{$V_{T}$}   	
	\label{gm_eq}
\end{equation}

\par For both stages, input and output impedances were also computed and the results are shown in the following table.

\begin{table}[ht]
\parbox{.45\linewidth}{
  \centering
  \begin{tabular}{|l|r|}
    \hline    
    {\bf Calculus} & {\bf Value [V]} \\ \hline
    \input{../sim/Zin_TAB}
  \end{tabular}
  \caption{Results for the voltage regulator. All variables are expressed in Volt. (Ngspice)}} 
\parbox{.45\linewidth}{
  \centering
  \begin{tabular}{|l|r|}
    \hline    
    {\bf Calculus} & {\bf Value [V]} \\ \hline
    \input{../sim/op_ZO_TAB}
  \end{tabular}
  \caption{Results for the voltage regulator. All variables are expressed in Volt. (Ngspice)}} 
\parbox{.45\linewidth}{
  \centering
  \begin{tabular}{|l|r|}
    \hline    
    {\bf Name} & {\bf Value [A or V]} \\ \hline
    \input{../mat/Z_TAB}
  \end{tabular}
  \caption{Results for the voltage regulator. All variables are expressed in Volt.(Octave)}}
\end{table}
 
 EXPLICAR DIFERENCAS DAS IMPEDANCIASSSSSSSSSSSSSSSSSSSSSSSSSSSSS
 
 Aditionally, the gain, bandwidth and cut off frequency results were also computed and compared.
 
 \begin{table}[ht]
\parbox{.45\linewidth}{
  \centering
  \begin{tabular}{|l|r|}
    \hline    
    {\bf Calculus} & {\bf Value} \\ \hline
    Gain& 34.0105\\ \hline
Central Frequency& 794.328\\ \hline
Gain deviation&49.8206\\ \hline
Central frequency deviation&205.672\\ \hline

  \end{tabular}
  \caption{Gain, bandwidth and cut off frequency. (Ngspice)}} 
\parbox{.45\linewidth}{
 \centering
  \begin{tabular}{|l|r|}
    \hline    
    {\bf Name} & {\bf Value} \\ \hline
    \input{../mat/r_theo_TAB}
  \end{tabular}
  \caption{Gain, bandwidth and cut off frequency. (Octave)}}
\end{table}
 
 
 EPLICAR DIFERENCAAAAAAAAAAAAAAAAAAAS
 







\section{Comparison}
\label{section:comparison}

\par In this section, a global comparison between Octave and Ngspice results will be made. Firstly, one should bear in mind that, in the previous sections, an Operating Point Analysis was performed to find the necessary values for the Incremental Analysis. 

\par As requested we are able to compute the passband frequency in simulation analysis using the measure funtion and the central frequency in the theoretical analysis using Low and High Band Pass Frequencies. The different calculation methods leads the results to a little difference.


 \begin{table}[h]
\parbox{.45\linewidth}{
  \centering
  \begin{tabular}{|l|r|}
    \hline    
    {\bf Calculus} & {\bf Value} \\ \hline
    V Gain&36.3869\\ \hline
Bandwidth&1.27868E+06\\ \hline
Lower Cut Off Freq& 53.0031\\ \hline

  \end{tabular}
  \caption{Central frequency [Hz] and respective gain [dB]. (Ngspice)}} 
\parbox{.45\linewidth}{
 \centering
  \begin{tabular}{|l|r|}
    \hline    
    {\bf Name} & {\bf Value} \\ \hline
    \input{../mat/wo_freq_gain_TAB}
  \end{tabular}
  \caption{Central frequency [Hz] and respective gain [dB]. (Octave)}}
\end{table}

We also computed the impedances as shown in the table bellow.


\begin{table}[ht]
\parbox{.45\linewidth}{
  \centering
  \begin{tabular}{|l|r|}
    \hline    
    {\bf Calculus} & {\bf Value [Ohm]} \\ \hline
    Zin & -738.418 + 125.718 j\\ \hline

  \end{tabular}
  \caption{Cirtcuit impedances. Variables are expressed in Ohm.(Ngspice)}} 
\parbox{.45\linewidth}{
 \centering
  \begin{tabular}{|l|r|}
    \hline    
    {\bf Name} & {\bf Value [Ohm]} \\ \hline
    \input{../mat/impedances_TAB}
  \end{tabular}
  \caption{Cirtcuit impedances. Variables are expressed in Ohm.(Octave)}}
\end{table}









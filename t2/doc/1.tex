\section{NODAL ANALYSIS}


\subsection{Theoretical Analysis}


In this section, a theoretical analysis of the circuit was conducted. The node method was the chosen approach.

The aim of using this method is to determine every node voltage. To do so, the node 4 was considered a reference node. Then, eight independent equations were written in orther to find the remaining unknown node voltage values. Before $t=0s$, v(s) is constant. Therefore, the capacitor behaves like an open-circuit which means Ix=0.

\begin{figure}[ht] \centering
\includegraphics[width=0.8\linewidth]{sim1draw.pdf}
\caption{Circuit analysed theoretically.}
\label{simdraw}
\end{figure}

The equations were then put in the form of the matrix shown below. Octave math tools were used to solve the system.\\

$\begin{bmatrix}
1 & 0 & 0 & 0 & 0 & 0 & 0 & 0\\
-G1 & G1+G2+G3 & -G2 & 0 & -G3 & 0 & 0 & 0\\
0 &-Kb-G2 & G2 & 0 & Kb & 0 & 0 & 0\\
0 & 0 & 0 & 1 & 0 & 0 & 0 & 0\\
0 & -G3 & 0 & -G4 & G3+G4+G5 & -G5 & -G7 & G7\\
0 & Kb & 0 & 0 & -Kb-G5 & G5 & 0 & 0\\
0 & 0 & 0 & -G6 & 0 & 0 & G6+G7 & -G7\\
0 & 0 & 0 & Kd*G6 & -1 & 0 & -Kd*G6 & 1\\
\end{bmatrix}
$

$\begin{bmatrix}
V1 \\ V2 \\ V3 \\ V4 \\ V5 \\ V6 \\ V7 \\ V8
\end{bmatrix}$
=
$\begin{bmatrix}
Vs \\ 0 \\ 0 \\ 0 \\ 0 \\ 0 \\ 0 \\ 0
\end{bmatrix}
$




\subsection{Operating Point Analysis}
First of all, to contextualize the values obtained using the tools in ngspice, it is necessary to state that, as node 0 is connected to ground, its nodal voltage does not appear on the table of results. Furthermore, to be able to describe the voltage flowing in the dependent source, it is necessary to know the current in resistor 6. However, ngspice is not able to compute this value when the depedent source is described. So, in order to do that, an extra dependent voltage source (whose voltage drop is equal to 0 V) was created, and put in series after the resistor 6. This led to the appearence of node 8, that has the same voltage drop as node 6. So, by doing that, ngspice is able to determine the current in this auxiliar independent source, which is exactly the value needed.
 The circuit with these changes is shown in the drawing below.

\begin{figure}[ht] \centering
\includegraphics[width=0.8\linewidth]{sim1draw.pdf}
\caption{Circuit analysed in ngspice.}
\label{simdraw}
\end{figure}

\subsection{Comparison}
After running the simulation, the results were put in the table below. Then, a careful analysis of the aforementioned table was conducted.It shows the simulated operating point results for the circuit that is being studied, allowing the group to obtain the current flowing in every risistor, the voltage in the dependent voltage source and even the current flowing in the dependent currrent source. 

A variable preceded by @ is of type {\em current} and expressed in Ampere; other variables are of type {\it voltage} and expressed in
    Volt.
\begin{table}[ht]
\parbox{.45\linewidth}{
  \centering
  \begin{tabular}{|l|r|}
    \hline    
    {\bf Name} & {\bf Value [A or V]} \\ \hline
    @cb[i] & 0.000000e+00\\ \hline
@ci[i] & 0.000000e+00\\ \hline
@cout[i] & 0.000000e+00\\ \hline
@q1[ib] & 2.132105e-05\\ \hline
@q1[ic] & 3.676881e-03\\ \hline
@q1[ie] & -3.69820e-03\\ \hline
@q1[is] & 1.945288e-12\\ \hline
@q2[ib] & 3.261073e-04\\ \hline
@q2[ic] & 4.602851e-02\\ \hline
@q2[ie] & -4.63546e-02\\ \hline
@q2[is] & -1.94801e-12\\ \hline
@r1[i] & 5.473928e-04\\ \hline
@r2[i] & 5.260718e-04\\ \hline
@rc[i] & 3.350773e-03\\ \hline
@re[i] & 3.698202e-03\\ \hline
@rin[i] & 0.000000e+00\\ \hline
@rl[i] & 0.000000e+00\\ \hline
@rout[i] & -4.63546e-02\\ \hline
base & 1.052144e+00\\ \hline
coll & 1.947680e+00\\ \hline
emit & 3.698202e-01\\ \hline
emit2 & 2.729077e+00\\ \hline
in & 0.000000e+00\\ \hline
in2 & 0.000000e+00\\ \hline
out & 0.000000e+00\\ \hline
vbe & 6.823234e-01\\ \hline
vcc & 1.200000e+01\\ \hline
vce & 1.577860e+00\\ \hline
veb & 7.813967e-01\\ \hline
vec & 2.729077e+00\\ \hline

  \end{tabular}
  \caption{Simulation and Calculus of Req (NgSpice)}} 
\parbox{.45\linewidth}{
  \centering
  \begin{tabular}{|l|r|}
    \hline    
    {\bf Name} & {\bf Value [A or V]} \\ \hline
    V1 & 5.068716e+00 \\ \hline
V2 & 4.843672e+00 \\ \hline
V3 & 4.369060e+00 \\ \hline
V4 & 0.000000e+00 \\ \hline
V5 & 4.874693e+00 \\ \hline
V6 & 5.579017e+00 \\ \hline
V7 & -1.980764e+00 \\ \hline
V8 & -2.974577e+00 \\ \hline

  \end{tabular}
  \caption{Simulation and Calculus of Req (NgSpice)}}
 
\end{table}


In order to validate the results obtained in NGSPICE, relative errors between the theoretical values, obtained in octave and the ones obtained in ngspice, were calculated. These were put in the table below.








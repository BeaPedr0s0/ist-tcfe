\section{Octave and Ngspice Comparison} \label{section:comp}

\subsection{Analysis for t$<$0}

After running the simulation, the results were put in the table below. Then, a careful analysis of the aforementioned table was conducted.It shows the simulated operating point results for the circuit that is being studied, allowing the group to obtain the current flowing in every risistor, the voltage in the dependent voltage source and even the current flowing in the dependent currrent source. 

A variable preceded by @ is of type {\em current} and expressed in Ampere; other variables are of type {\it voltage} and expressed in
    Volt.
\begin{table}[ht]
\parbox{.45\linewidth}{
  \centering
  \begin{tabular}{|l|r|}
    \hline    
    {\bf Name} & {\bf Value [A or V]} \\ \hline
    @cb[i] & 0.000000e+00\\ \hline
@ci[i] & 0.000000e+00\\ \hline
@cout[i] & 0.000000e+00\\ \hline
@q1[ib] & 2.132105e-05\\ \hline
@q1[ic] & 3.676881e-03\\ \hline
@q1[ie] & -3.69820e-03\\ \hline
@q1[is] & 1.945288e-12\\ \hline
@q2[ib] & 3.261073e-04\\ \hline
@q2[ic] & 4.602851e-02\\ \hline
@q2[ie] & -4.63546e-02\\ \hline
@q2[is] & -1.94801e-12\\ \hline
@r1[i] & 5.473928e-04\\ \hline
@r2[i] & 5.260718e-04\\ \hline
@rc[i] & 3.350773e-03\\ \hline
@re[i] & 3.698202e-03\\ \hline
@rin[i] & 0.000000e+00\\ \hline
@rl[i] & 0.000000e+00\\ \hline
@rout[i] & -4.63546e-02\\ \hline
base & 1.052144e+00\\ \hline
coll & 1.947680e+00\\ \hline
emit & 3.698202e-01\\ \hline
emit2 & 2.729077e+00\\ \hline
in & 0.000000e+00\\ \hline
in2 & 0.000000e+00\\ \hline
out & 0.000000e+00\\ \hline
vbe & 6.823234e-01\\ \hline
vcc & 1.200000e+01\\ \hline
vce & 1.577860e+00\\ \hline
veb & 7.813967e-01\\ \hline
vec & 2.729077e+00\\ \hline

  \end{tabular}
  \caption{Simulation nodal voltage results. All variables are expressed in Volt or Ampere. (Ngspice)}} 
\parbox{.45\linewidth}{
  \centering
  \begin{tabular}{|l|r|}
    \hline    
    {\bf Name} & {\bf Value [A or V]} \\ \hline
    V1 & 5.068716e+00 \\ \hline
V2 & 4.843672e+00 \\ \hline
V3 & 4.369060e+00 \\ \hline
V4 & 0.000000e+00 \\ \hline
V5 & 4.874693e+00 \\ \hline
V6 & 5.579017e+00 \\ \hline
V7 & -1.980764e+00 \\ \hline
V8 & -2.974577e+00 \\ \hline

  \end{tabular}
  \caption{Theoretical nodal voltage results. All variables are expressed in Volt.(Octave)}}
 
\end{table}


In order to validate the results obtained in NGSPICE, relative errors between the theoretical values, obtained in octave and the ones obtained in ngspice, were calculated. These were put in the table below.



After running the node analysis in octave and the simulation in ngspice, the results were put in the tables below. Then, a careful analysis of the aforementioned tables was conducted.


\clearpage
\subsection{Calculus and Simulation of $R_{eq}$.}

A variable preceded by @ is of type {\em current} and expressed in Ampere; other variables are of type {\it voltage} and expressed in
    Volt. $R_{eq}$ is presented as in \ref{eq:4} in Ohm.

\begin{table}[ht]
\parbox{.45\linewidth}{
  \centering
  \begin{tabular}{|l|r|}
    \hline    
    {\bf Name} & {\bf Value [A or V]} \\ \hline
    @gb[i] & -4.13942e-18\\ \hline
@r1[i] & -3.95801e-18\\ \hline
@r2[i] & -4.13942e-18\\ \hline
@r3[i] & 1.814124e-19\\ \hline
@r4[i] & 8.624201e-19\\ \hline
@r5[i] & -2.74542e-03\\ \hline
@r6[i] & -8.67362e-19\\ \hline
@r7[i] & -1.72875e-18\\ \hline
v(1) & 0.000000e+00\\ \hline
v(2) & -4.12073e-15\\ \hline
v(3) & -1.28112e-14\\ \hline
v(5) & -3.55271e-15\\ \hline
v(6) & 8.553593e+00\\ \hline
v(7) & 1.776352e-15\\ \hline
v(8) & 3.552714e-15\\ \hline
v(9) & 1.776352e-15\\ \hline

  \end{tabular}
  \caption{Simulation of $R_{eq}$. All variables are expressed in Volt, Ohm or Ampere. (NgSpice)}} 
\parbox{.45\linewidth}{
  \centering
  \begin{tabular}{|l|r|}
    \hline    
    {\bf Name} & {\bf Value [A or V]} \\ \hline
    V1 & 0.000000e+00 \\ \hline
V2 & 0.000000e+00 \\ \hline
V3 & 9.496396e-16 \\ \hline
V4 & 0.000000e+00 \\ \hline
V5 & 5.935248e-17 \\ \hline
V6 & 8.553593e+00 \\ \hline
V7 & -2.967624e-17 \\ \hline
V8 & 0.000000e+00 \\ \hline
Ix & -2.745419e-03 \\ \hline
Req & 3.115588e+03 \\ \hline
tau & 3.244185e-03 \\ \hline

  \end{tabular}
  \caption{Calculus of $R_{eq}$.All variables are expressed in Volt, Ohm or Ampere. (Octave)}}
 
\end{table}

\par After the careful evaluation of the results, several observations need to be made. Firstly, in ngspice, there are some node voltage results different from 0V, with  values  in the order of magnitude of $10^-15$, $10^-16$ V. After questioning the professor, these will be considered 0V.

\par On the other hand, octave results have equally those discrepancies. That said, the procedure was the same and those values were considered 0. We believe the main reason for this situation is that, when the data.txt file is read from the datagen.py in octave, a rounding happens, making it impossible for the nodal voltages to be 0. 

\par However, and despite the small erros above explaned, the values of $R_{eq}$, $V_{6}$ and $I_{x}$ match perfectly, which led us to fully validate the theorical procedure and the results obtained.



























